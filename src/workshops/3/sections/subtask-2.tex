\section{Delopgave 2}\label{sec:delopgave-23}

\subsection{Bevis, at \textsc{MergeSort} kan sortere lister med heltal}
\label{subsec:bevis-at-mergesort-kan-sortere-lister-med-heltal}

\subsubsection{Basistrin}

Lad \(n = 1 \rightarrow l = 0, r = 0\).
\(L\) bliver returneret, uden at blive modificeret, men er også sorteret, fordi der kun er ét element i listen.

\subsubsection{Induktionstrin}

Induktionshypotese: \textsc{MergeSort} anvendes på en delliste med længde \(l < n\) og returnerer den samme liste
sorteret.

Lad \(n > 1\).
Vi kalder \textsc{MergeSort} på hele intervallet \lstinline{MergeSort(L, 0, n - 1)}.
Intern vil \textsc{MergeSort} så kalde \textsc{MergeSort} igen på \lstinline{MergeSort(L, 0, floor((n - 1) / 2)))} og
\lstinline{MergeSort(L, floor((n - 1) / 2) + 1, n - 1)}.
Efter induktionshypotesen vil de to kald til \textsc{MergeSort} returnere sorterede lister efter ``opgaveantagelsen'',
dvs. \textsc{Merge} returnere en sammenflettet, sorteret liste.
