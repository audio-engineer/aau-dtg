\section{Delopgave 3}\label{sec:delopgave-33}

\subsection{\textsc{Merge} med \(m + n - 1\) sammenligninger}\label{subsec:merge-med-m-+-n-1-sammenligninger}

\subsubsection{Basistrin (k = 0)}

Lad \(k = 0\), \(n = 2^0 = 1\).
Når \(n = 1\) element er kun én sammenligning nødvendig, nemlig \lstinline{l < r}.

\begin{equation}
    P: 2^k(k + 1) = 2^0(0 + 1) = 1
    \label{eq:equation18}
\end{equation}

\subsubsection{Induktionstrin}

Lad os bevise den for \(k \rightarrow k + 1\).

\(n = 2^{k + 1} = 2 \cdot 2^k\).

Vi deler listen i to dele, men fordi listens længde altid er potenser af \(2\), kommer dellisterne altid have samme
længde på hver side af træet.

Ifølge induktionshypotesen har \textsc{MergeSort} \(2^k(k + 1)\) sammenligninger på en delliste med længde \(l = 2^k\).
\textsc{Merge} har \(2^k + 2^k - 1\) sammenligninger.

\begin{equation}
    \begin{aligned}
        S_{k + 1} & = 1 + 2^k(k + 1) + 2^k(k + 1) + 2^k + 2^k - 1 \\
        & = 2^k(k + 1) + 2^k(k + 1) + 2^k + 2^k \\
        & = 2 \cdot 2^k(k + 1) + 2 \cdot 2^k \\
        & = 2^{k + 1}(k + 1) + 2^{k + 1} \\
        & = 2^{k + 1}(k + 2)
    \end{aligned}
    \label{eq:equation19}
\end{equation}

\subsection{\textsc{MergeSort} bruger mindre end \(2n\log_2(n)\) sammenligninger}
\label{subsec:merge-sort-bruger-mindre-end-sammenligninger}

\subsubsection{Basistrin (k = 0)}

Hvis \(n = 2\) bruger \textsc{MergeSort} \(4\) sammenligninger.

\begin{equation}
    \begin{aligned}
        2n\log_2(n) = 2 \cdot 2\log_2(2) = 4
    \end{aligned}
    \label{eq:equation20}
\end{equation}

Hvis \(n = 3\) bruger \textsc{MergeSort} \(8\) sammenligninger.

\begin{equation}
    \begin{aligned}
        2n\log_2(n) = 2 \cdot 3\log_2(3) \approx 6 \cdot 1,585 = 9,51
    \end{aligned}
    \label{eq:equation21}
\end{equation}

Fordi \(8 \leq 9,51\) passer vores basistrin.

\subsubsection{Induktionstrin}

Lad os bevise den for \(n \rightarrow n + 1\).

\begin{equation}
    \begin{aligned}
        S_{n + 1} & = 1 + 2 \left\lfloor \frac{n + 1}{2} \right\rfloor \log_2\left( \left\lfloor \frac{n + 1}{2}
        \right\rfloor \right) + 2 \left\lceil \frac{n + 1}{2} \right\rceil \log_2\left( \left\lceil \frac{n + 1}{2}
        \right\rceil \right) n + 1 - 1 \\
        & \leq n + 1 + 2 \log_2\left( \left\lceil \frac{n + 1}{2} \right\rceil \right)\left( \left\lfloor \frac{n +
        1}{2} \right\rfloor + \left\lceil \frac{n + 1}{2} \right\rceil \right) \\
        & = n + 1 + 2\log_2\left( \left\lceil \frac{n + 1}{2} \right\rceil \right)(n + 1) \\
        & = (n + 1) \left(2\log_2\left( \left\lceil \frac{n + 1}{2} \right\rceil \right) + 1 \right) \\
        & \leq (n + 1) \left( 2\log_2\left( \frac{2}{3}(n + 1) \right) + 1 \right) \\
        & = (n + 1) \cdot 2 \log_2(n + 1) - (n + 1) \cdot \log_2\left( \frac{3}{2} \right) + n + 1 \\
        & \leq 2(n + 1) \log_2(n + 1)
    \end{aligned}
    \label{eq:equation22}
\end{equation}

\subsection{Hvilken slags induktion brugte vi?}\label{subsec:hvilken-slags-induktion-brugte-vi?}

Stærk induktion.
