\section{Delopgave 1}\label{sec:delopgave-12}

\subsection{Hvorfor er \(f\) ikke en matematisk funktion?}\label{subsec:hvorfor-er-(f)-ikke-en-matematisk-funktion?}

Funktionen for \(f\) kan beskrives sådan:

\begin{equation}
    f((a_{1}, a_{2}, a_{3} \dots a_{n}), x)
    \label{eq:equation12}
\end{equation}

Det betyder at det array-funktionen returnerer er indexet som \(x\) eksisterer på.
Definitionen af en funktion lyder: \(f: A \rightarrow B\) er en tildeling af præcis et element i \(B\) for hvert element
i \(A\).

Hvis vi bruger den definition på den givne array \(\left(3, 4, 4, 5\right)\) observerer vi at ved \(f(4)\) vil få to
værdier for \(f(x)\) nemlig \(f(4) = 1 \wedge 2\), men dette bryder definition ovenfor og derfor er det ikke en
funktion.

\subsection{Er \(f\) injektiv, surjektiv eller bijektiv?}\label{subsec:er-(f)-injektiv-surjektiv-eller-bijektiv?}

Funktionen er ikke injektiv fordi at, ud fra defininition fra injektivitet må \( f: A \rightarrow B \) maks antage en
værdi når den går fra \(A\) til \(B\), men når vi kigger efter værdier ude fra mængden vil funktionen antage værdien
\(0\).

Dette udelukker også at den er bijektiv da det kræver at den både er, surjektiv og injektiv.

Surjektivitet er defineret ved at funktionen mindst skal antage en værdi for alle A til B\@.
Det opfylder funktionen da vi har at hvis ikke \(A_{i}\) eksisterer i mængden antager værdien \(0\).
