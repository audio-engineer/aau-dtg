\section{Delopgave 2}\label{sec:delopgave-22}

Funktionen er ikke injektiv fordi at, ud fra defininition fra injektivitet må \( f: A \rightarrow B \) maks antage en
værdi når den går fra \(A\) til \(B\), men når vi kigger efter værdier ude fra mængden vil funktionen antage værdien
\(0\).

Dette udelukker også at den er bijektiv da det kræver at den både er, surjektiv og injektiv.

Surjektivitet er defineret ved at funktionen mindst skal antage en værdi for alle A til B\@.
Det opfylder funktionen da vi har at hvis ikke \(A_{i}\) eksisterer i mængden antager værdien \(0\).
