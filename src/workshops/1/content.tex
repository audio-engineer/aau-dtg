\section{Workshop 1}\label{sec:workshop-1}

\import{./sections/}{subtask-1.tex}

\subsection{Delopgave 2}\label{subsec:delopgave-2}

\subsubsection{Hvad svarer de to muligheder til?}\label{subsubsec:hvad-svarer-de-to-muligheder-til?}

\subsubsection*{Mulighed 1}
Mulighed 1 er en sammensætning af potensrelationer.

\begin{equation}
    \begin{split}
        R^2 = R \cup \{\,(\text{Gry}, \text{Noa}), (\text{Mia}, \text{Noa}), \\
        (\text{Kim}, \text{Liv}), (\text{Kim}, \text{Tom}), (\text{Mia}, \text{Tom})\,\}
    \end{split}\label{eq:equation7}
\end{equation}

Fordi \(R\) er refleksiv, kan vi danne en sammensætning af \(R\)

\subsubsection*{Mulighed 2}
Transitiv aflukning, fordi vi laver sammensætninger af alle potenser af relationen.

\subsubsection{Hvilken aflukning er der tale om?}
Der er tale om en transitiv aflukning, som kan beskrives på følgende måde:

\begin{equation}
    S = \bigcup_{i=1}^{7} \, R^{i} = R \cup R^2 \cup R^3 \cup \ldots \cup R^7 \label{eq:equation6}
\end{equation}

Men reelt set kan man også beskrive aflukningen som \(S = R^{3}\) fordi \(R^4\) ikke tilføjer nye par til
relationen, osv.

\subsection{Delopgave 3}\label{subsec:delopgave-3}

\subsubsection{Hvilke elementer indholder mængderne?}

Mængden \(F_{\text{Tom}}\) er en mængde bestående af elementer \(a\), som er alle elementer i par i \(S\), som
\(\text{Tom}\) har en forbindelse til:

\begin{equation}
    \begin{split}
        F_{\text{Tom}} = \{\,\text{Tom}, \text{Kim}, \text{Gry}, \\
        \text{Noa}, \text{Liv}, \text{Mia}\,\}
    \end{split}\label{eq:equation9}
\end{equation}

På samme måde kan vi også beskrive \(F_{\text{Noa}}\):

\begin{equation}
    F_{\text{Noa}} = \{\,\text{Noa}, \text{Mia}, \text{Gry}, \text{Liv}, \text{Kim}\,\}\label{eq:equation8}
\end{equation}

\(F_{\text{Tom}} \cap F_{\text{Noa}}\) er fællesmængden (intersection) af \(F_{\text{Tom}}\) og \(F_{\text{Noa}}\)
og indholder alle elementer som \(F_{\text{Tom}}\) og \(F_{\text{Noa}}\) har tilfælles:

\begin{equation}
    F_{\text{Tom}} \cap F_{\text{Noa}} = \{\,\text{Noa}, \text{Liv}, \text{Mia}, \text{Gry}, \text{Kim}\,\}
    \label{eq:equation10}
\end{equation}

\subsubsection{Vis, at \(G_{\text{Tom}} \subseteq G_{\text{Tom}}\)}

Lad \(B \subseteq A, G_{\text{Tom}} = \{\,b \in B \mid (b, \text{Tom}) \in S\,\}\label{eq:equation11}\).
Lad \(p \in G_{\text{Tom}} \implies p \in B \implies \textcolor{purple}{p \in A}\).
Lad \(p \in G_{\text{Tom}} \implies \textcolor{teal}{(p, \text{Tom}) \in S}\).

Derfor følger:

\begin{equation}
    F_{\text{Tom}} = \{\,\textcolor{purple}{a \in A} \mid \textcolor{teal}{(a, \text{Tom}) \in S}\,\} \implies
    G_{\text{Tom}} \subseteq F_{\text{Tom}}\label{eq:equation2}
\end{equation}
