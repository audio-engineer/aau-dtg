\section{Delopgave 2}\label{sec:delopgave-2}

\subsection{Hvad svarer de to muligheder til?}\label{subsec:hvad-svarer-de-to-muligheder-til?}

\subsection*{Mulighed 1}

Mulighed 1 er en sammensætning af potensrelationer.

\begin{equation}
    \begin{split}
        R^2 = R \cup \{\,(\text{Gry}, \text{Noa}), (\text{Mia}, \text{Noa}), \\
        (\text{Kim}, \text{Liv}), (\text{Kim}, \text{Tom}), (\text{Mia}, \text{Tom})\,\}
    \end{split}\label{eq:equation7}
\end{equation}

Fordi \(R\) er refleksiv, kan vi danne en sammensætning af \(R\) \subsection*{Mulighed 2} Transitiv aflukning, fordi vi
laver sammensætninger af alle potenser af relationen.

\subsection{Hvilken aflukning er der tale om?}\label{subsec:hvilken-aflukning-er-der-tale-om?}
Der er tale om en transitiv aflukning, som kan beskrives på følgende måde:

\begin{equation}
    S = \bigcup_{i=1}^{7} \, R^{i} = R \cup R^2 \cup R^3 \cup \ldots \cup R^7 \label{eq:equation6}
\end{equation}

Men reelt set kan man også beskrive aflukningen som \(S = R^{3}\) fordi \(R^4\) ikke tilføjer nye par til relationen,
osv.
